% !TEX root = ../Projektdokumentation.tex
\section{Implementierung}
\label{sec:implementierung}

Der Autor hat eine mögliche Implementierung der \schulze in Java entwickelt. Die untenstehende Implementierung arbeitet nach der ''winning votes'' Methode, die in Beispiel 3 in Abschnitt \ref{sec:winningVotes} beschrieben wird. Wie in Beispiel 3 (Abschnitt: \ref{sec:beispiel3}) zu erkennen ist, gibt es auch für die anderen Möglichkeiten einen Sieger zu bestimmen.

Eine Vollständige Implementierung kann man auf Gitub finden.\footnote{\url{https://github.com/SteffenHo/SchulzeImplementation}} Dort sind auch Beispielaufrufe hinterlegt. 

\lstinputlisting[
language=Java,
numbers=none,
caption={Grundlegender Funktionsaufbau der \schulze in Java},
captionpos=b]
{Listings/Schulze.java}

