% !TEX root = ../Projektdokumentation.tex
\newpage
\section{Bewertung des Algorithmus}
\label{sec:Bewertung2}

\subsection{Überführung}
\label{sec:Ueberfuehrung}
Im Folgenden wird kurz erläutert, wie die \schulze Schritt für Schritt in einen Algorithmus überführt wird.

Als Eingabe wird die Menge $N$ als zweidimensionales Array dem Algorithmus übergeben. Dort sind die Stimmen eingetragen, die jeder Kandidat gegen jeden anderen Kandidaten bekommen hat.

Als Ausgabe erhält man ein Array der Sieger.

\subsubsection{Initialisierung}
\label{init}
Zuerst muss die Menge $P_{D}$ initialisiert werden, dies ist ein zweidimensionales Array, dass die schwächste Verbindungen der stärksten Wege aufnimmt. Dort werden initial auch die Werte aus der Menge $N$ übernommen, abhängig vom Ansatz $margin$, $ratio$, $winning votes$ und $loosing votes$, die in Beispiel 3 erläutert wurden.

\subsubsection{Stärkste Wege}
\label{starkeWege}
Nun müssen die Stärksten Wege gefunden werden. Dazu müssen drei Schleifen ineinander verschachtelt werden. Die jeweils alle Kandidaten durchgehen, also jeweils $C$ mal läuft. Nun werden alle Wege zwischen zwei Kandidaten gesucht, dazu dienen zwei Schleifen und eine dritte Schleife sorgt dafür, dass alle Wege untersucht werden. Wird ein Weg gefunden, der stärker ist, als der Weg der bisher eingetragen ist so wird der bestehende Wert mit dem Wert des stärkeren Wegs überschrieben.  

Nachdem die Auswertung durchgelaufen ist, enthält das Arras $P_{D}$ für jedes Duell jeweils die schwächste Verbindung der stärksten Wege.

\subsubsection{Ergebnis}
Nun muss der Sieger ermittelt werden. Dazu wird ein neues Array winner erstellt und jeder Kandidat wird dort erstmal als Sieger eingetragen. Es werden nun wieder mit zwei Schleifen alle Kandidaten mit allen verglichen. Wenn ein Kandidat gegen seinen Gegenkandidaten verliert, wird der Kandidat in dem winner Array sofort auf false gesetzt, da er nicht mehr gewinnen kann, da er einmal geschlagen wurde. Nachdem alle Duelle stattgefunden haben enthält das Array winner $C$ Werte, das zurückgegeben wird. $true$ bedeutet, dass der Kandidat ein Sieger ist. Der Kandidat wird über den Index des Arrays identifiziert.


% !TEX root = ../Projektdokumentation.tex
\section{Implementierung}
\label{sec:implementierung}

Der Autor hat eine mögliche Implementierung der \schulze in Java entwickelt. Die untenstehende Implementierung arbeitet nach der ''winning votes'' Methode, die in Beispiel 3 in Abschnitt \ref{sec:winningVotes} beschrieben wird. Wie in Beispiel 3 (Abschnitt: \ref{sec:beispiel3}) zu erkennen ist, gibt es auch für die anderen Möglichkeiten einen Sieger zu bestimmen.

Eine Vollständige Implementierung kann man auf Gitub finden.\footnote{\url{https://github.com/SteffenHo/SchulzeImplementation}} Dort sind auch Beispielaufrufe hinterlegt. 

\lstinputlisting[language=JAva,numbers=none]{Listings/Schulze.java}

\subsection{Laufzeit}
\label{sec:Laufzeit}
Die Laufzeitkomplexität der \schulze ist mit $O(C^3)$ nicht besonders gut, wobei C die Anzahl der Kandidaten ist. Diese Komplexität sieht man auch in der Implementierung in Abschnitt \ref{sec:implementierung}, dort wird über mit einer dreifach verschachtelten Schleife über ein Array iteriert. Jedoch ist die Laufzeit zu relativieren, da diese Methode zum Auswerten einer Wahl nur einmal laufen muss, um ein Ergebnis zu liefern.

Auch wird die Zahl der Kandidaten meist nicht ins unendliche steigen, da es in normalen Wahlen eine endlich oft recht begrenzte Anzahl von Kandidaten gibt. Trotzdem muss man die Komplexität beachten, wenn man die \schulze in Systeme einbaut, die nicht eine klassische Abstimmung, wie es sie z.B. in Politik gibt, entsprechen und C beliebig groß werden kann.


