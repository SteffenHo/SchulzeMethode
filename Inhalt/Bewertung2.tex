% !TEX root = ../Projektdokumentation.tex
\section{Bewertung des Algorithmus}
\label{sec:Bewertung2}

Die Laufzeitkomplexität der \schulze ist mit $O(C^3)$ nicht besonders gut, wobei C die Anzahl der Kandidaten ist. Diese Komplexität sieht man auch in der Implementierung in Abschnitt \ref{sec:implementierung}, dort wird über mit einer dreifach verschachtelten Schleife über ein Array iteriert. Jedoch ist die Laufzeit zu relativieren, da diese Methode zum auswerten einer Wahl nur einmal laufen muss, um ein Ergebnis zu liefern.

Auch wird die Zahl der Kandidaten meist nicht ins unendliche steigen, da es in normalen Wahlen eine endlich oft recht begrenzte Anzahl von Kandidaten gibt. Trotzdem muss man die Komplexität beachten, wenn man die \schulze in Systeme einbaut, die nicht einer klassische Abstimmung, wie es sie z.B. in Politik gibt, entsprechen und C beliebig groß werden kann.


