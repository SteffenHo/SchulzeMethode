% !TEX root = ../Projektdokumentation.tex
\section{Einleitung}
\label{sec:Einleitung}


\subsection{Markus Schulze} 
\label{sec:markusSchulze}
Die \schulze wurde nach seinem Erfinder Markus Schulze benannt und wird in Fachkreisen auch als \glqq Schwartz Sequential dropping\grqq{} oder auch \glqq path winner\grqq{} Methode bezeichnet.

Die Schulze Methode ist ein verfahren, um eine Wahl auszuwerten und im besten Fall einen eindeutigen Sieger zu finden.

Er hat diese Methode erstmals 1997 in einer offenen Mail zur Diskussion gestellt \citep{Schulze1997} und veröffentliche immer wieder aktualisierte Versionen seiner Theorie. In dieser Ausarbeitung bezieht sich der Autor dabei auf seine aktualisierten Ausarbeitungen aus den Jahren 2017 und 2018. .

\footcite[vgl.]{Schulze2018}

\subsection{Problemstellung} 
\label{sec:problemstellung}
In einer Demokratie, aber auch in größeren Gruppen ist es oft sehr schwer eine Wahl gerecht auszuwerten, um aus den Präferenzen des einzelnen einen Kompromiss zu finden, der für alle maximal Zufriedenstellen ist.

Dieses Problem einen gerechten Sieger zu finden, soll mit der \schulze gelöst werden. Die \schulze fällt in das Gebiet der Sozialwahltheorie. Dieses interdisziplinäre Forschungsgebiet beschäftigt sich mit der Untersuchung von Gruppenentscheidungen. In dieser Forschung werden individuelle Präferenzen und Entscheidungen der der teilnehmenden Personen aggregiert, um daraus eine \glqq gerechte\grqq{} kollektive Entscheidung abzuleiten. Damit man eine \glqq gerechte\grqq{} Methode finden kann, beschäftigen sich viele Teilbereiche der Forschung, wie z.B. die Mathematik, Volkswirtschaft, Psychologie, Philosophie, Politikwissenschaft und Rechtswissenschaft mit diesem Thema und stellen dabei verschiedenste Ansätze und Vorgehensweisen vor. Alle beteiligten Forschungsgebiete stellen dabei Definitionen auf, was eine \glqq gerechte\grqq{} Methode erfüllen soll. \citep{scheubrein2013computerunterstuetzte}

Die \schulze ist eine condorcete Wahlmethoden, dass bedeutet das immer zwei Kandidaten verglichen werden und ein Sieger aus diesem Vergleich hervorgeht und aus denen dann ein Gesamtsieger gefunden wird. 

Es haben sich über die Jahre viele Qualitätskriterien entwickelt, an denen man messen kann, ob eine Methode im Sinne der Sozialwahltheorie gerechte ist.

Im Folgenden werden einige Kriterien definiert, die in der Sozialwahltheorie von Bedeutung sind. In Abschnitt \ref{sec:Bewertung1} werden diese Kriterien erneut untersucht und festgestellt in wieweit die Schulze Methode diesen Kriterien gerecht wird. Viele dieser Kriterien gelten für Methoden, die einen Sieger oder mehrere Sieger ermitteln. 

\subsubsection{Monotonie Kriterium} 
\label{sec:monotoniekriterium}
Der Gewinner einer Wahl kann nicht durch ein besseres Ranking verlieren und ein Verlierer durch ein schlechteres gewinnen. \citep{Woodall1996} Dieses Kriterium ist trivial, aber zeigt, dass die zugrundeliegende Wahlmethode nicht der erwarteten Logik einer Wahl widerspricht.

\subsubsection{\condorcet Kriterium} 
\label{sec:condorectKriterium}
Nach der Wahl wird ein Zweikampf zweier Kandidaten Simuliert und dabei untersucht, wie oft der Kandidat $a$ dem Kandidat $b$ vorgezogen wurde. Die Bedeutung von \glqq vorgezogen\grqq{} wir in Abschnitt \ref{sec:beispiel3} für die \schulze erläutert, ist im allgemeinen aber Abhängig von der gewählten Wahlmethode. \condorcetSieger ist der Kandidat der alle anderen Kandidaten Schlägt. Einen solchen Sieger muss es nicht geben. Ein Wahlverfahren erfüllt das Condorect-Kriterium, wenn der gewählte Sieger auch der \condorcetSieger ist, sofern es einen \condorcetSieger gibt. \citep{Johnson2005}

\subsubsection{Lösbarkeits Kriterium} 
\label{sec:loesbarkeitsKriterium}
Eine Wahlmethode erfüllt dieses Kriterium, wenn es einen eindeutigen Sieger gibt, hierbei gibt es zwei Ansätze dies zu prüfen \citep{Schulze2017}
\begin{enumerate}
\item Wenn die Menge der Stimmen Richtung unendlich tendiert, geht der die Wahrscheinlichkeit keinen eindeutigen Sieger zu haben gegen Null.
\item Wenn es mehr als einen Sieger gibt, reicht einen einzelnen Stimme, um einen eindeutigen Sieger zu erhalten.
\end{enumerate}
Dieses Kriterium wird wichtig, wenn die Duelle der Kandidaten simuliert werden. Sofern Kandidaten gleich bewertet werden können wird Punkt zwei wichtig.

\subsubsection{Pareto Kriterium} 
\label{sec:paretoKriterium}
Dieses Kriterium gibt an, dass
\begin{enumerate}
\item wenn jeder Wähler Alternative $a$, Alternative $b$ vorzieht, muss Alternative $a$ immer Alternative $b$ bevorzugt werden
\item wenn kein Wähler Alternative $a$, Alternative $b$ vorzieht, muss Alternative $a$ nicht besser sein als $b$. \citep{Schulze2017}
\end{enumerate}

Dies ist nur ein Ausschnitt von Kriterien, die für eine gerechte Wahlmethode gelten sollen. Eine größere Übersicht kann in Markus Schulzes \glqq The Schulze Method
of Voting\grqq{} \citep{Schulze2018} Kapitel 11 gefunden werden. Dort wird in Kapitel 4 auch bewiesen, dass die \schulze die Kriterien erfüllt. Ein Beispiel für die Richtigkeit des \condorcet Kriteriums kann auch im ersten Beispiel (Abschnitt: \ref{sec:beispiel1})  gefunden werden.



