% !TEX root = ../Projektdokumentation.tex
\section{Einleitung}
\label{sec:Einleitung}


\subsection{Markus Schulze} 
\label{sec:markusSchulze}
Die Schulze Methode wurde nach seinem Erfinder Markus Schulze benannt und wird in Fachkreisen auch als "Schwartz Sequential dropping" oder auch "path winner" Methode bezeichnet.

Die Schulze Methode ist ein verfahren, um aus einer Liste von Kandidaten einen eindeutigen Sieger zu ermitteln.

Er hat diese Methode zuerst 1997 erstmal in einer offenen Mail zur Diskussion gestellt und veröffentliche immer wieder aktualisierte Versionen seiner Theorie. In dieser Ausarbeitung bezieht sich der Autor dabei auf seine aktualisierte Ausarbeitung aus dem Jahr 2017.\citet[vgl.]{Schulze2017}.

\subsection{Problemstellung} 
\label{sec:problemstellung}
Welches Problem soll diese Methode lösen?

\subsection{Anforderungen} 
\label{sec:anforderungen}
Welche Anforderungen werden an einen solchen Algorithmus gestellt.





