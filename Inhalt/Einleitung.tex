% !TEX root = ../Projektdokumentation.tex
\section{Einleitung}
\label{sec:Einleitung}


\subsection{Markus Schulze} 
\label{sec:markusSchulze}
Die \schulze wurde nach ihrem Erfinder Markus Schulze benannt und wird in Fachkreisen auch als \glqq Schwartz Sequential Dropping\grqq{} oder auch \glqq path winner\grqq{} Methode bezeichnet. Markus Schulze ist ein Mathematiker und hat diese Methode an der TU Berlin entwickelt.

Die Schulze Methode ist ein Verfahren, um eine Wahl auszuwerten und im besten Fall einen eindeutigen Sieger zu Finden.

Er hat diese Methode erstmals 1997 in einer offenen E-Mail zur Diskussion gestellt\footnote{\Vgl \citet{Schulze1997}.} und veröffentlicht immer wieder aktualisierte Versionen seiner Theorie. In dieser Ausarbeitung werden als Primärquellen  die aktualisierten Ausarbeitungen aus den Jahren 2017\footnote{\Vgl \citet{Schulze2017}} und 2018\footnote{\Vgl \citet{Schulze2018}} genutzt.

\subsection{Problemstellung} 
\label{sec:problemstellungBeispiel}
Zu Beginn soll ein Beispiel zeigen, wie schwer es sein kann einen gerechten Sieger zu finden. Ein Kurs von 30 Personen soll aus drei Kandidaten einen Kurssprecher wählen. Jede Person bringt die drei Kandidaten in eine Rangfolge ( Erst-, Zweit- und Drittwunsch). Diese Art der Wahl kann mit der \schulze ausgewertet werden. 

Dort hat sich folgende Stimmsituation ergeben, 10 Wähler  haben die Reihenfolge $a$, $b$, $c$, 10 Wähler die Reihenfolge $b$, $a$, $c$ und 10 Wähler die Reihenfolge $c$, $a$, $b$ gewählt.

Bei Betrachtung dieses Beispiels fällt auf, dass kein Kandidat im direkten Duell seine beiden Gegner schlagen kann. Kein Kandidat ist von allen Wählern direkt bevorzugt worden. Naiv kann man sagen, dass man Kandidat $a$ nimmt, da dieser Kandidat 10-mal Erstwunsch und 20-mal Zweitwunsch ist, dass wäre besser als die Ergebnisse der anderen Kandidaten. 

Ist das eine gerechte Entscheidung? Bei einer überschaubaren Menge von 30 Personen könnte darüber abgestimmt werden, ob das gerecht ist. Bei einer Wahl von mehreren Hundert oder Tausend Wählern wird dies schon schwieriger. Dort wird es wichtig eine Methode zu wählen, die als \glqq gerecht\grqq{} gilt.
Kriterien, um zu definieren was gerecht ist, wird vom Forschungsgebiet der Sozialwahltheorie untersucht.

\newpage

\subsection{Sozialwahltheorie} 
\label{sec:problemstellung}
Die Einführungssituation hat schon gezeigt, dass es sehr schwer ist eine Wahl gerecht auszuwerten. In einer  Demokratie wird es eine komplexe Herausforderung aus den Präferenzen der einzelnen Wähler einen Kompromiss zu finden, der für alle maximal zufriedenstellend ist.

Ein Ansatz einen gerechten Sieger zu finden, soll mit der \schulze gegeben werden. Die \schulze fällt in das Gebiet der Sozialwahltheorie. Dieses interdisziplinäre Forschungsgebiet beschäftigt sich mit der Untersuchung von Gruppenentscheidungen. In dieser Forschung werden individuelle Präferenzen und Entscheidungen der teilnehmenden Personen aggregiert, um daraus eine \glqq gerechte\grqq{} kollektive Entscheidung abzuleiten. Damit man eine \glqq gerechte\grqq{} Methode finden kann, beschäftigen sich viele Teilbereiche der Forschung, wie z.B. die Mathematik, Volkswirtschaft, Psychologie, Philosophie, Politikwissenschaft und Rechtswissenschaft mit diesem Thema und stellen dabei verschiedenste Ansätze und Vorgehensweisen vor. Alle beteiligten Forschungsgebiete stellen dabei Kriterien auf, welche eine \glqq gerechte\grqq{} Methode erfüllen soll\footnote{\Vgl \citet{scheubrein2013computerunterstuetzte} Seite 91 ff.}.

Die \schulze ist eine Abwandlung der \condorcet Wahlmethode, dass bedeutet das immer zwei Kandidaten verglichen werden und ein Sieger aus diesem Vergleich hervorgeht. Es haben sich über die Jahre viele Qualitätskriterien entwickelt, an denen man messen kann, ob eine Methode im Sinne der Sozialwahltheorie als gerecht betrachtet werden kann. In der abschließenden Bewertung werden einige dieser Kriterien betrachtet.





