% !TEX root = ../Projektdokumentation.tex
\section{Einleitung}
\label{sec:Einleitung}


\subsection{Markus Schulze} 
\label{sec:markusSchulze}
Die Schulze Methode wurde nach seinem Erfinder Markus Schulze benannt und wird in Fachkreisen auch als \glqq Schwartz Sequential dropping\grqq{} oder auch \glqq path winner\grqq{} Methode bezeichnet.

Die Schulze Methode ist ein verfahren, um aus einer Liste von Kandidaten einen eindeutigen Sieger zu ermitteln.

Er hat diese Methode zuerst 1997 erstmal in einer offenen Mail zur Diskussion gestellt und veröffentliche immer wieder aktualisierte Versionen seiner Theorie. In dieser Ausarbeitung bezieht sich der Autor dabei auf seine aktualisierte Ausarbeitung aus dem Jahr 2017.\citet[vgl.]{Schulze2017}.

\subsection{Problemstellung} 
\label{sec:problemstellung}
Das Problem einen eindeutigen Sieger zu finden, das mit der vorgestellten Theorie gelöst werden soll, fällt in das Gebiet der Sozialwahltheorie. Dieses interdisziplinäre Forschungsgebiet beschäftigt sich mit der Untersuchung von Gruppenentscheidungen. In dieser Forschung werden individuelle Präferenzen und Entscheidungen der der teilnehmenden Personen aggregiert, um daraus eine \glqq gerechte\grqq{} kollektive Entscheidung abzuleiten. Damit man eine \glqq gerechte\grqq{} Methode finden kann, beschäftigen sich viele Teilbereiche der Forschung, wie z.B. die Mathematik, Volkswirtschaft, Psychologie, Philosophie, Politikwissenschaft und Rechtswissenschaft mit diesem Thema und stellen dabei verschiedenste Ansätze und Vorgehensweisen vor. Alle beteiligten Forschungsgebiete stellen dabei Definitionen auf, was eine \glqq gerechte\grqq{} Methode erfüllen muss. \cite[vgl.]{scheubrein2013computerunterstuetzte}

Daher haben sich über die Jahre Qualitätskriterien entwickelt, an denen man Messen kann, ob eine Methode im Sinne der Sozialwahltheorie \glqq gerechte\grqq{} ist.

Im Folgenden werden einige Kriterien definiert, die in der Sozialwahltheorie von Bedeutung sind. In Abschnitt XX werden diese Kriterien erneut untersucht und festgestellt in wieweit die Schulze Methode gerecht ist. Viele dieser Kriterien gelten für Methoden, die einen Sieger oder mehrere Sieger ermitteln. Da die Schulze Methode, eine Methode ist, um einen Sieger zu ermitteln, werden die Definitionen auf diese Eigenschaft eingegrenzt.

\subsubsection{Monotonie Kriterium} 
\label{sec:monotoniekriterium}
Der Gewinner einer Wahl kann nicht durch ein besseres Ranking verlieren und ein Verlierer durch ein schlechteres gewinnen. \cite{Woodall1996}

\subsubsection{Condorect Kriterium} 
\label{sec:condorectKriterium}
Nach der Wahl wird ein zweikampf zweier Kandidaten Simuliert und dabei untersucht, wie oft der Kandidat A dem Kandidat B vorgezogen wurde. Condorect-Sieger ist der Kandidat der alle anderen Kandidaten Schlägt. Einen solchen Sieger muss es nicht geben. Ein Wahlverfahren erfüllt das Condorect-Kriterium, wenn der gewählte Sieger auch der Condorect-Sieger ist, sofern es einen Condorect-Sieger gibt. \cite{Johnson2005}

\subsubsection{Lösbarkeits Kriterium} 
\label{sec:loesbarkeitsKriterium}
Eine Wahlmethode erfüllt dieses Kriterium, wenn es einen Eindeutigen Sieger gibt, hierbei gibt es zwei Ansätze dies zu prüfen \citet{Schulze2017}
\begin{enumerate}
\item Wenn die Menge der Stimmen Richtung unendlich tendiert, geht der die Wahrscheinlichkeit keinen eindeutigen Sieger zu haben gegen Null 
\item Wenn es mehr als einen Sieger gibt, reicht einen einzelnen Stimme, um einen eindeutigen Sieger zu erhalten.
\end{enumerate}

\subsubsection{Pareto Kriterium} 
\label{sec:paretoKriterium}
Dieses Kriterium gibt an, dass
\begin{enumerate}
\item wenn jeder Wähler Alternative A, Alternative B vorzieht, muss Alternative A immer Alternative B bevorzugt werden
\item wenn kein Wähler Alternative A, Alternative B vorzieht, muss Alternative A nicht besser sein als B. \citet{Schulze2017}
\end{enumerate}

\subsubsection{LIIA} 
\label{sec:anforderungen}
Welche Anforderungen werden an einen solchen Algorithmus gestellt.

\subsubsection{Smith} 
\label{sec:anforderungen}
Welche Anforderungen werden an einen solchen Algorithmus gestellt.

\subsubsection{Prudence} 
\label{sec:anforderungen}
Welche Anforderungen werden an einen solchen Algorithmus gestellt.

\subsubsection{MinMax Set} 
\label{sec:anforderungen}
Welche Anforderungen werden an einen solchen Algorithmus gestellt.

\subsubsection{Schwartz} 
\label{sec:anforderungen}
Welche Anforderungen werden an einen solchen Algorithmus gestellt.

\subsubsection{Participation} 
\label{sec:anforderungen}
Welche Anforderungen werden an einen solchen Algorithmus gestellt.

\subsubsection{Reversal Symmetry} 
\label{sec:anforderungen}
Welche Anforderungen werden an einen solchen Algorithmus gestellt.v




