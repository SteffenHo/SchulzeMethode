% !TEX root = ../Projektdokumentation.tex
\section{Bewertung der Methode}
\label{sec:Bewertung1}

Die Nachfolgende Tabelle \ref{schulzeKramer} zeigt die \schulze im Vergleich zur Simpson-Kramer Methode, die in Abschnitt \ref{sec:alternativeAlgorithmen} erläutert wird. Sie zeigt in welchen Kriterien die \schulze und die Simpson-Kramer Methode Kriterien der Sozialwahltheorie erfüllen. Ein Ausschnitt der Bedingungen wurde in Abschnitt \ref{sec:problemstellung} beschrieben, eine vollständige Ausführung aller Kriterien kann in der Ausarbeitung von Martin Schulze nachgelesen werden \citep{Schulze2018}.

% !TEX root = Projektdokumentation.tex

\begin{longtable}[c]{|l|l|l|}
\hline
Kriterien &Schulze                       & Simpson-Kramer     \\ \hline
\endfirsthead
%
\endhead
%
resolvability                 & Ja             & Ja   \\ \hline
Pareto                        & Ja             & Ja   \\ \hline
reversal symmetry             & Ja             & Nein \\ \hline
monotonicity                  & Ja             & Ja   \\ \hline
independence of clones        & Ja             & Nein \\ \hline
Smith                         & Ja             & Nein \\ \hline
Smith-IIA                     & Ja             & Nein \\ \hline
Condorcet                     & Ja             & Ja   \\ \hline
Condorcet loser               & Ja             & Nein \\ \hline
majority for solid coalitions & Ja             & Nein \\ \hline
majority                      & Ja             & Ja   \\ \hline
majority loser                & Ja             & Nein \\ \hline
participation                 & Nein           & Nein \\ \hline
MinMax set                    & Ja             & Nein \\ \hline
prudence                      & Ja             & Ja   \\ \hline
polynomial runtime            & Ja             & Ja   \\ \hline
\caption{Vergleich der \schulze mit der Simpson-Kramer Methode.}
\label{schulzeKramer}\\
\end{longtable}

Bei der Tabelle wir deutlich, dass es auch die \schulze nicht schafft alle Kriterien zu erfüllen jedoch vielen Anforderungen bestand hält, die andere Methoden nicht erfüllen. Und daher deutlich wird, das die \schulze eine gute \condorcet Methode ist um einen Sieger zu ermitteln.
\newpage

\subsection{Eindeutigkeit}
\label{sec:eindeutigkeit}
Die \schulze wurde als eine Methode vorgestellt, die einen Sieger ermittelt. Jedoch ist das nicht ganz richtig. Es kann auch vorkommen, dass zwei Sieger gefunden werden. Wenn man eine Situation wie diese

\begin{description}
\centering
\item[3 mal] $a \succ_{v} b \succ_{v} c \succ_{v}d$
\item[2 mal] $c \succ_{v} b \succ_{v} d \succ_{v}a$
\item[2 mal] $d \succ_{v} a \succ_{v} b \succ_{v}c$
\item[2 mal] $d \succ_{v} b \succ_{v} c \succ_{v}a$
\end{description} 
untersucht, erhält man nicht einen Sieger sondern zwei, $\mathcal{S}=\{b,d\}$. Die \schulze bietet keine Möglichkeit einen eindeutigen Sieger in dieser Situation zu ermitteln. In diesen Fällen wird geraten eine Stichwahl zwischen den Kandidaten zu machen.
