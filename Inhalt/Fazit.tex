% !TEX root = ../Projektdokumentation.tex
\newpage
\section{Fazit}
\label{sec:Fazit}

\subsection{Eingangsbeispiel}
Ganz zu Beginn (Abschnitt \ref{sec:problemstellungBeispiel}) wurde eine Wahlsituation von 30 Wählern und 3 Kandidaten beschrieben, die auf den ersten Blick kein Sieger geliefert hat. Die Wahl kann mit der \schulze ausgewertet werden und liefert dann Kandidat $a$ als Sieger. Dies ist auch der selbe Sieger der ersten Idee, $a$ zu nehmen weil der Kandidat auch viele Zweitstimmen hatte. Dies hat sich gerecht angefühlt, jedoch ohne mathematischen Beweis. Die \schulze kann mathematisch beweisen, dass sie nach bestimmten Kriterien gerecht ist. Daher bietet sich diese Methode immer an, wenn die Lösung nicht trivial ist oder die Menge der Wahlberechtigten groß ist. 

\subsection{Einsatz} 
\label{sec:einsatz}
Erstmals wurde die \schulze 2003 im Debian, eine Linux Distribution, eingesetzt. Dort waren es ca. 1000 Wahlberechtigte, die ihre Wahlen mit dieser Methode auswerten.\footnote{\Vgl \citet{Debian2018}} Sie nutzen die \schulze z.B. um bestimmte Features auszuwählen. Beispielsweise wurde im Jahr 2014 mit der \schulze über das Init-System für Debian abgestimmt.\footnote{\Vgl \citet{Leemhu2014}}



2008, 2009 und 2011 wurde die \schulze von Wikimedia, den Dachverband von Wikipedia, genutzt, um zu entscheiden wer die Führung der Organisation übernehmen soll. Es waren im Jahr 2011 43.000 Wahlberechtigte. \footnote{\Vgl \citet{Schulze2017}}

Auch in der Politik hat diese Methode ihre Heimat gefunden. 2009 hat die Piratenpartei von Schweden diese Wahlmethode eingeführt, 2010 die Piratenpartei Deutschland, 2011 die Australische Piratenpartei, 2013 die Piratenpartei Island, 2015 die niederländische Piratenpartei.\footnote{\Vgl \citet{Lohmann2013}}

Die \schulze hat sich über die Jahre zu der am weitesten verbreiteten \condorcet Methode entwickelt. Über 60 Organisationen mit über 800.000 Wahlberechtigten nutzen diese Methode.\footnote{\Vgl \citet{Schulze2018}} Des weiteren arbeiten auch einige online Tools damit, wie z.B. GoogleVotes, bei denen man nicht genau beziffern kann, wie viele Wahlberechtigte diese Tools generieren.\footnote{\Vgl \citet{Hardt2015}}







