% !TEX root = ../Projektdokumentation.tex
\section{Fazit}
\label{sec:Fazit}

% !TEX root = ../Projektdokumentation.tex
\subsection{Vergleich mit anderen Methoden}
\label{sec:alternativeAlgorithmen}

Das Forschungsgebiet der Sozialwahltheorie ist seit dem 19. Jahrhundert ein wichtiges Gebiet, um gerechte Wahlen zu garantieren. Daher gibt es auch eine große Anzahl von anderen Methoden, die einen Sieger hervorbringen. Eine Methode die laut Barry Wright\footnote{\Vgl \citet{Wright2009}} mit hoher Wahrscheinlichkeit das selbe Ergebnis liefert, wie die \schulze ist die Simpson-Kramer Methode.

Diese Methode erklärt die Person zum Sieger, deren größte Niederlage kleiner war als alle Niederlagen der anderen Kandidaten\footnote{\Vgl \citet{Nurmi2017}}.  

Wie in Tabelle \ref{schulzeKramer} zu sehen ist, erfüllt diese Methode aufgrund seiner Simplizität nur eine geringe Anzahl von Kriterien im Vergleich zur \schulze. Diese Methode wird auch als Minmax-Methode bezeichnet.



\subsection{Einsatz} 
\label{sec:einsatz}
Erstmals wurde die \schulze 2003 im Debian, eine Linux Distribution, eingesetzt. Dort waren es ca. 1000 Wahlberechtigte, die nun ihre Wahlen mit dieser Methode auswerten. \citep{Debian2018}

2008, 2009 und 2011 wurde die \schulze von Wikimedia, den Dachverband von Wikipedia, genutzt, um zu entscheiden wer die Führung der Organisation übernehmen soll. Es waren in 2011 43.000 Wahlberechtigte. \citep{Schulze2017}

Auch in der Politik hat diese Methode ihre Heimat gefunden. 2009 hat die Piratenpartei von Schweden diese Wahlmethode eingefürht, 2010 die Piratenpartei Deutschland, 2011 die Australische Piratenpartei, 2013 die Piratenpartei Island, 2015 die niederländische Piratenpartei. \citep{Lohmann2013}

Die \schulze hat sich über die Jahre zu der am weitesten verbreiteten \condorcet Methode entwickelt. Über 60 Organisationen mit über 800.000 Wahlberechtigten nutzen diese Methode, genauso wie viele online Tools, wie GoogleVotes. \citep{Schulze2018}







