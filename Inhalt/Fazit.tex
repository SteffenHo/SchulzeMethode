% !TEX root = ../Projektdokumentation.tex
\section{Fazit}
\label{sec:Fazit}

\subsection{Einsatz} 
\label{sec:einsatz}
Erstmals wurde die \schulze 2003 im Debian, eine Linux Distribution, eingesetzt. Dort waren es ca. 1000 Wahlberechtigte, die nun ihre Wahlen mit dieser Methode auswerten. \citep{Debian2018}

2008, 2009 und 2011 wurde die \schulze von Wikimedia, den Dachverband von Wikipedia, genutzt, um zu entscheiden wer die Führung der Organisation übernehmen soll. Es waren in 2011 43.000 Wahlberechtigte. \citep{Schulze2017}

Auch in der Politik hat diese Methode ihre Heimat gefunden. 2009 hat die Piratenpartei von Schweden diese Wahlmethode eingefürht, 2010 die Piratenpartei Deutschland, 2011 die Australische Piratenpartei, 2013 die Piratenpartei Island, 2015 die niederländische Piratenpartei. \citep{Lohmann2013}

Die \schulze hat sich über die Jahre zu der am weitesten verbreiteten \condorcet Methode entwickelt. Über 60 Organisationen mit über 800.000 Wahlberechtigten nutzen diese Methode, genauso wie viele online Tools, wie GoogleVotes. \citep{Schulze2018}







